\documentclass{article}
\usepackage{fullpage}

\usepackage[ngerman, english]{babel}
\renewcommand{\familydefault}{\sfdefault}
\usepackage[scaled=1]{helvet}
\usepackage[helvet]{sfmath}
\everymath={\sf}

\usepackage{parskip}
\usepackage[colorinlistoftodos]{todonotes}
\usepackage[colorlinks=true, allcolors=blue]{hyperref}
\usepackage{siunitx}
\DeclareSIUnit\molar{M}

\title{Manual for the LabView Integration UI}
\author{Johann Brenner}
%\company{The University of Western Ontario}
\setcounter{tocdepth}{3}

\newcounter{ListCounter}


\begin{document}
	\maketitle
	\tableofcontents 



\section{User Guide}

\subsection{Prequisites}
	If the software is not already installed on the computer, refer to the README.md file.

\subsubsection*{What NOT to do}
\begin{itemize}
	\item	The program is not very fast. In case something does not happen instantly, be patient and avoid clicking multiple times. This mostly results in unpredictable behavior.
	\item Never, I repeat, \textbf{never} use the \textbf{Close}-Button (On Windows $\rightarrow$ the X) on the top right corner of any window. This will crash the program in every case. The same applies to the intrinsic \textit{LabVIEW-Abort Program} button near the top left corner. Always use the special \textbf{Abort}. \textbf{Finish} or \textbf{Stop} buttons inside the UI.
	\item Normally, the COM ports can stay unmodified. However, (mainly after Windows Updates) these ports tend to change. Their correctness can be verified in the Device-Manager (Geräte-Manager) by plugging them out and back in.
	\item If a device does not turn green upon program start, do not worry. This happens (unfortunately) quite often. Just click the big \textbf{STOP} Button and restart. If the device is still not recognized try switching the respecive and/or your computer off and back on again.
	\item Do not open a .tif file, while an acquisition is running. 
\end{itemize}


\subsection{Starting the software}
\label{sec:software_start}
\begin{enumerate}
	\item Ensure that the following devices are booted and connected to the PC
	\begin{itemize}
		\item Olympus Microscope
		\item TANGO Stage Controller
		\item SOLA Light Engine
		\item Valve Controller 
	\end{itemize}
	\item Start the LabvView project UI\_{}Project.lvproj
	\item Expand the VIs subfolder and start the latest version of the UI
	\item Click the Run or Run-Continously button on the top left corner
	\item The red lamps on the top-left corner turn green upon successful connection to the respective device.
	\subitem If a lamp stays red:
	\begin{itemize}
		\item Hit the STOP-Button and start over
		\item If the lamp is still red, switch the respective device off and on again.
		\item Ensure the correct COM-Port is selected. (Via Device-Manager or NI-Device Monitor)
	\end{itemize}

	\item Wait until the status field says ``Initialization finished successfully''
	\item 	Now, the program is fully operational.
\end{enumerate}



\subsection{Operating a Multiring chip}
Make sure you started the program according to section \ref{sec:software_start}.\\
A general remark: You can capture and modify the positions as well as the microscope lists with the buttons in their proximity. To select a specific configuration, double click on the desired row. A single click will only select the row for a modification by \textbf{Replace} or \textbf{Clear Selected}. 

	\subsubsection{Initalization}
	\begin{enumerate}
	\item Start the Elveflow device and the \textbf{ESI} software
	\item Disable all 4 valves in the software
	\item Pressurize the Elveflow with \SIrange{4}{8}{\bar} by turning the right valve (blue) between the pressure regulators above the microscope
	\item Wipe the microfluidic top and bottom with Acetone, IPA and ddH20 to remove dust etc. (otherwise it will cause imaging artifacts)
	\item 	Fill the control tubing at least (!) \SI{20}{\centi\meter} with ddH2O with a \SI{1}{\milli\liter} syringe and connect it to the respective inlet
	\item 	Insert the chip carefully into the microscope
	\item 	Ensure that the tubing does not interfere with the other devices, corners or edges and can move freely during stage driving. Move a sufficient length into the microscope to avoid pulling at the inlets.
	\item Connect a \SI{40}{\micro\molar} Fluorescein solution to the Elveflow
	\begin{itemize}
		\item Use the small diameter PTFE tubing for the whole length.
		\item Apply \SIrange{10}{20}{\milli\bar} at the respective \textbf{ESI} channel to move the fluid to the end of the tubing until a drop starts to form
		\item  Insert the connector to Inlet 1 of the chip
	\end{itemize}

	\item Connect ddH$_{2}$O or nfH$_{2}$O to the Elveflow		\begin{itemize}
		\item Apply \SIrange{10}{20}{\milli\bar} at the respective \textbf{ESI} channel to move the fluid to the end of the tubing until a drop starts to form
		\item  Insert the connector to Inlet 2 of the chip
	\end{itemize}
	\item Build a connector from small-diameter PTFE-tubing to big-diameter PTFE-tubing; connect it to a waste container and the outlet of the chip (To minimize the hydraulic resistance and capacitive air effects)
	\item Start the Labview program
	\item Click the \textbf{Live} Button, then adjust \textbf{Filter}, \textbf{Intensity} and \textbf{Exposure} settings.
	\item Open the controle pressure with the left, blue valve on top of the microscope and pressurize the control valves with approx. \SI{1.5}{\bar} by turning the center and left pressure regulator.
	\item Click the \textbf{No Flow} Button to pressurize all quake valves
	\setcounter{ListCounter}{\value{enumi}}
		\end{enumerate}
	
	\subsubsection{Loading the chip}
		\begin{enumerate}
			\setcounter{enumi}{\value{ListCounter}}
	\item Check the control inlets for leaks and wait for the channels to fill, control it in the Live mode. Now would be a great time for experienced users to save the different calibration positions.
	\item Pressurize the \textbf{ESI} channels for water and fluorescein to \SI{250}{\milli\bar}
	\item Select \textit{Reagent 1} and click the \textbf{Flush} Button to inject the fluorescein
		\setcounter{ListCounter}{\value{enumi}}
	\end{enumerate}
		
	\subsubsection{Adjusting the valve pressure}
	\begin{enumerate}                 
	\setcounter{enumi}{\value{ListCounter}}
	\item Upon arrival, deactivate \textbf{Flush} and switch to the \textit{GFP} filter; adjust \textit{exposure} (exit live mode) and \textit{intensity} as well as the histogram sliders
	\item Use the \textit{10x Objective} and move to a valve in the reagent-multiplexer.
	\subitem Increase the pressure until you see two meniski, which indicate that the valve closes completely.
	\item Repeat the same procedure for the ring pump	\setcounter{ListCounter}{\value{enumi}}
\end{enumerate}


	
	\subsubsection{Remove air}
			\begin{enumerate}
		\setcounter{enumi}{\value{ListCounter}}
	\item select \textit{Reagent 2} and click the \textbf{Flush} Button again to inject the water
	\subitem For faster flushing, the \textbf{ESI} pressure can be increased \textbf{temporarily} up to \SI{500}{\milli \bar}  
	\subitem Upon arrival, let it flush as long as air is in the flushing channel between the rings
	\item Deactivate the \textbf{Flush}, click \textbf{All Valves} and load all rings with \textbf{Load}
	\subitem Wait until all rings are partly filled
	\subitem Activate manually the valve \textit{A5} to load another side of the ring
	\subitem Deactivate \textit{A5} and open \textit{B5} to close the outlet.
	\subitem This will pressurize all channel and remove remaining air bubbles.
	\subitem Test the RemoveAir button. It should do the same.
		\setcounter{ListCounter}{\value{enumi}}
\end{enumerate}


	
	\subsubsection{Performing the pump calibration}
	\begin{enumerate}
		\setcounter{enumi}{\value{ListCounter}}
	\item Click the \textbf{No Flow} Button
	\item Ensure the \textit{Position List} is empty
	\subitem if not, hit the \textbf{Clear Last} button
	\item Move to every ring, starting at the bottom (of live mode) and capture one position of each ring with \textbf{Save Pos}. (The bottom most ring corresponds to RingNr. 1)
	For this step, the 4x objective is sufficient. However, if you want to use the calibration pictures as reference for your measurement, you should use the same settings as in your later program.
	\item Exit \textit{Live Mode}, wait some time and hit \textbf{Capture Blank} of the calibration panel at the bottom; follow the instructions and wait until the acquisition is finished
	\item Change to \textit{GFP}, select a desired \textit{intensity}, go into \textbf{Live} and \textbf{Flush} with Fluorescein for \SI{10}{\second}
	\item Make sure \textbf{All Valves} is still on and load all rings with \textbf{Load}
	\item Exit \textbf{Live} and \textbf{Capture Full Intensity}
	\item Go into \textbf{Live}, make sure \textbf{All Valves} is still on and \textbf{Load} all rings with Fluorescein
	\item Exit \textbf{Load} and adjust \textit{Feed} and \textit{Pump} cycles
	\subitem Best results were achieved with $>$800 \textit{Pump Cycles} and $\sim$30 \textit{Feed Cycles}
	\item Hit \textbf{Capture Dilutions}
	\subitem After the first dilution cycle, the (a bit under-)estimated remaining time is displayed 
	\subitem At finish, an input dialog with 8 refresh ratios is displayed
	\item Start MATLAB and run the \textit{CalibrationScript.m} under \textit{lib $\rightarrow$ Calib}
	\subitem Write the according \textbf{Refresh Ratios per Pump Cycle} back into Labview
	\subitem This is not ideal, I know...	\setcounter{ListCounter}{\value{enumi}}
\end{enumerate}


	\subsubsection{Creating a program}
	\begin{enumerate}
		\setcounter{enumi}{\value{ListCounter}}
	\item \textbf{Optionally: }Clear the \textit{Position List} and capture your desired positions.
	\subitem \textbf{NOTE:} As the focus is always saved with the position, it makes a difference if the positions are captured in \textit{4x}, \textit{10x} or another magnification
	\item Save your position list by \textbf{Save List} with the suffix \textit{.xml} in a desired location
	\item Clear the \textit{Microscope Settings} by clicking the \textbf{Clear} Button
	\item Choose your desired acquisition settings (\textit{Filter}, \textit{Exposure}, \textit{Intensity}) and \textbf{Save Settings}
	\subitem Repeat it for every color you want to acquire
	\item \textbf{Save List} with the suffix \textit{.xml} (preferably) at the same location as the positions
	\item Make sure that the Refresh Ratios are set correctly. This is crucial for any further step!
	\subitem If you want to change the Refresh Ratios, you can either double click into the respective ratio and change it or hit the \textbf{Input Refresh Ratios} to input all ratios again.
	\item Start the \textbf{Program Setup}
	\item If you wish to modify a previously created program, click on \textbf{Load Old Program}
	\item Select the previouly created \textit{Microscope} and \textit{Position Settings} by clicking the \textbf{Folder Symbol}
	\subitem If the lists have been loaded correctly, the LED beneath the Folder turns green
	\item Now you can modify the program by clicking the respective buttons:
	\subitem \textbf{Feed}: Exchanges a defined fraction of a single ring
	\subitem \textbf{Pump}: Mixes all rings by actuating the ring pump. $\left(\textsf{Formula:\ Time}=\frac{\textsf{Pump\ Cycles}}{\SI{4}{\hertz}}\right)$
	\subitem \textbf{Change Reagent}: Flushes the outer channels with the selected reagent for a defined time
	\subitem \textbf{Loop Start}: Begins a unique Loop with a defined number of iterations 
	\subitem \textbf{Loop End}: Closes a unique Loop with its specified ID
	\subitem \textbf{Acquire}: Acquires an image at all positions each with every microscope setting.
	\item Hit \textbf{Save} 
	\subitem For simple changes you can also modify an existing \textit{.xml} file accordingly.
	\item When program reaches the \textbf{Total Acquisition Time}, it terminates automatically
	\item Until the \textbf{Total Acquisition Time} is reached, the outermost loop is repeated every \textbf{Interval}. We mostly use \SI{15}{\minute}
	\item The later filename is composed of \textbf{YYMMDD\_{}hh\_{}mm\_{}ss\_{}''YOURPREFIX''.tif}
	\item All images which are captured during the program will be stored in the \textbf{Image Output Folder}. You can either select an existing directory by moving into the desired folder and selecting \textbf{Verzeichnis wählen} or move to another folder, type in the folder name and selecting \textbf{Save. }Make sure that there is enough free space on the drive.
	\item The previously created program will be saved in a \textit{.xml} file. Navigate to the desired location, type in the file name with its ending and hit \textbf{Save}. 

	
		\setcounter{ListCounter}{\value{enumi}}
\end{enumerate}
	\subsubsection{Starting a program}
\begin{enumerate}
\setcounter{enumi}{\value{ListCounter}}
\item Check if the refresh ratios are still there and valid
\item Hit the \textbf{Run Program} button
\item Select the previously created \textit{program.xml} file and hit \textbf{Load}
\item Check if all values and steps are correct
\item Start the continuous loop via the \textbf{Start} button
\end{enumerate}
After starting the Program, a queue with all program steps is initialized and the timing is computed. Please be patient. This may take some time.
\end{document}
